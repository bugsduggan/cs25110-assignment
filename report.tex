\documentclass[a4paper, twoside]{article}

\usepackage{fixltx2e}

\setlength{\oddsidemargin}{0in} \setlength{\evensidemargin}{0in}
\setlength{\textwidth}{6.2in}
\setlength{\topmargin}{-0.3in} \setlength{\textheight}{9.8in}
\setlength{\parskip}{0.3in}

\title{CS25110 - Scenario Analysis Report}
\author{Tom Leaman (thl5)}

\begin{document}
\maketitle
\newpage

\section{Introduction}
% 2000 Students, 50 Academic Staff, 50 Graduate/research, 10 Admin

\section{Server Room}
\subsection{Physical Requirements}
% Ceiling/floor height
% Rack space

\subsection{Service Provision Requirements}
% Power
% Network

\subsection{Other needs}
% Cooling

The server room must be fitted with a fire suppression system. This should use a
gasseos suppression agent such as CO\textsubscript{2} which will not damage the
sensitive equipment. This \emph{will}, however require more space to be made
available to store the suppression agent.

Physical access to the room must also be kept to only those staff members who
require access to complete their tasks (and every effort should be made to
enable them to work without access to the server room). Access will also be
monitored through the key card access system (see \ref{sec:security}).

\section{Project Management}

\section{Hardware}
% Application Server(s)
% NAS
% Firewall
% Network switches
% Print/mail server
% Second site switch

\section{Software}
\subsection{IT \& Services Software}
All of the servers will be installed with the most current version of Debian. I
believe this to be one of the most well-established and supported server OS
available (and it's free).

% Firewall
% E-Mail

\subsection{User Software}
All student and staff workstations will be installed with Windows 7. I Believe
this to be a well understood platform for the average user and it may be
possible to purchase a support contract from Microsoft.

Research students may require other versions of Windows or even Linux to be
installed in order to run legacy applications critical to their work. In this
case all software should pass through the audit procedures outlined below (see
\ref{sec:security}) before it is installed.

% Office package (LibreOffice)

\section{Security}
\label{sec:security}
% Software audit
% Regular maintainence
% Server room access

\section{Naming \& Directory Services}

\section{Business Continuity}
% What are the critical functions?

\section{Ongoing Management}
% Logs
% Maintainence schedule
% Backups
% Training (rehearsal of disaster plans)

\section{Summary}

\section{Bibliography}

\end{document}
