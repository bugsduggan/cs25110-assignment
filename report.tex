\documentclass[a4paper, twoside]{article}

\usepackage{fixltx2e}

\setlength{\oddsidemargin}{0in} \setlength{\evensidemargin}{0in}
\setlength{\textwidth}{6.2in}
\setlength{\topmargin}{-0.3in} \setlength{\textheight}{9.8in}
\setlength{\parskip}{0.3in}

\title{CS25110 - Scenario Analysis Report}
\author{Tom Leaman (thl5)}

\begin{document}
\maketitle
\newpage
\tableofcontents
\newpage

\section{Introduction}
% 2000 Students, 50 Academic Staff, 50 Graduate/research, 10 Admin

\section{Server Room}
\subsection{Physical Requirements}
% Ceiling/floor height

There will be 3 42U server racks in the centre of the room allowing full access
to both the front and the back of the cabinet. It is imperative that these be
grounded correctly so that any electrical anomalies do not present a threat to
human health. These will provide adequate space for all the equipment specified
in this document in addition to space for future requirements.

\subsection{Service Provision Requirements}
Power will be provided to the server room by way of a big grey box on the wall.
% TODO
This will be distributed to the hardware by way of rack-attached power strips
with monitoring capabilities (such as those provided by Geist).

This power will be backed-up by an Uninterruptable Power Supply. % TODO

% Network

\subsection{Other needs}
Server cooling will be provided by 3 Airedale OnRak units. These will dissipate
the heat generated and will fit into the back of each of the server racks.

The server room must be fitted with a fire suppression system. This should use a
gasseous suppression agent such as CO\textsubscript{2} which will not damage the
sensitive equipment. This \emph{will}, however require more space to be made
available to store the suppression agent.

Physical access to the room must also be kept to only those staff members who
require access to complete their tasks (and every effort should be made to
enable them to work without access to the server room). Access will also be
monitored through the key card access system (see \ref{sec:security}).

\section{Project Management}
\label{sec:projman}

\section{Hardware}
% Application Server(s)
% NAS (HP X9000)
% Firewall
% Network switches
% Print/mail server
% Second site switch

\section{Software}
\subsection{IT \& Services Software}
All of the servers (with the exception of the firewall and network storage covered below) will be installed with the most current version of Debian. I
believe this to be one of the most well-established and supported server OS
available (and it's free).

\subsubsection{Application Servers}
The application servers will be used to provide software as a service to the
users. The following list is expected to grow as a result of further
requirements gathering activities outlined in section \ref{sec:projman}

Intacct Financial and Management Accounting software will be installed to
provide payroll, PoS, budgeting and accounting services to administrative staff
and public cafe/bar areas.

A Learning Management System will be provided by Blackboard. Blackboard has a
large market share and industry backing \cite{bb} and although open-source
alternatives were considered, none provide the range of features and support
available from Blackboard.

\subsubsection{Firewall}
One of the servers will operate as a dedicated firewall and gateway. It will be
installed with Smoothwall, a dedicated firewall distribution of Linux. It is
open-source and fairly well supported through its online documentation and
manuals.

\subsubsection{E-mail}
One of the servers will function as a dedicated e-mail server. It will be
installed with Exim as the Mail Transfer Agent and RoundCube as the webmail
front-end. Exim has been growing in popularity over recent years and is well
supported on Debian platforms\cite{exim}.

\subsubsection{Printing}
Print services will be provided using the CUPS set of standard print spooling
tools on another, lower powered, Debian machine.

\subsubsection{Network Storage}
The network storage specified comes with its own specialised OS and software.
There is a client application available to access the storage but since the
device uses standard network file storage protocols, this should not be
necessary (and is not available for Debian systems at the time of writing).

\subsubsection{Webserver}
One server will be given over to providing both the external website and an
internal intranet. Nginx will be installed as the webserver. Nginx has been
growing in popularity since its inception \cite{nginx}. It is open-source and considerable
less memory-intensive than Apache.

\subsection{User Software}
All student and staff workstations will be installed with Windows 7. I Believe
this to be a well understood platform for the average user and it may be
possible to purchase a support contract from Microsoft.

Research students may require other versions of Windows or even Linux to be
installed in order to run legacy applications critical to their work. In this
case all software should pass through the audit procedures outlined below (see
\ref{sec:security}) before it is installed.

General office tools (word processor, spreadsheets, presentations etc.) will be
provided by LibreOffice. This is not too dissimilar from commercial offerings
and does not require licenses to be purchased.

\section{Security}
\label{sec:security}
% Software audit
% Regular maintenance

\section{Naming \& Directory Services}

\section{Business Continuity}
% What are the critical functions?

\section{Ongoing Management}
% Logs
% Maintenance schedule
% Backups
% Training (rehearsal of disaster plans)

\section{Summary}

\begin{thebibliography}{1}

\bibitem{rack} {\em http://www.iblparts.co.uk/forsale/5B933FE8.html}
\bibitem{fire} {\em
  http://www.techrepublic.com/blog/security/the-mystical-world-of-data-center-fire-suppression/4113}
\bibitem{bb} {\em http://en.wikipedia.org/wiki/Learning\_management\_system}
\bibitem{smoothwall} {\em http://www.smoothwall.org/}
\bibitem{exim} {\em http://www.securityspace.com/s\_survey/data/man.201007/mxsurvey.html}
\bibitem{nginx} {\em http://trends.builtwith.com/Web-Server/nginx}

\end{thebibliography}

\end{document}
