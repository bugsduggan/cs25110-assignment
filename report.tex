\documentclass[a4paper, twoside]{article}

\usepackage{fixltx2e}

\setlength{\oddsidemargin}{0in} \setlength{\evensidemargin}{0in}
\setlength{\textwidth}{6.2in}
\setlength{\topmargin}{-0.3in} \setlength{\textheight}{9.8in}

\title{CS25110 - Scenario Analysis Report}
\author{Tom Leaman (thl5)}

\begin{document}
\maketitle
\newpage
\tableofcontents
\newpage

\section{Introduction}
This document provides an overview of the ICT systems inteded to be provided for
Mid Wales University. The University has around 2000 students and it is assumed
to have around 50 academic staff, 50 post-graduate / research staff and 10
administrative staff.

Mid Wales University consists of two campuses in Newton and Caersws. One of
these is assumed to provide office facilities for the majority of the
administrative activities and shall be the location of the server room
(henceforth considered to be the main campus). There will also be a second
collection of servers on the second site providing redundancy for the main
server room. The two sites will be connected by a fiber optic link discussed in
section \ref{sec:servicepro}.

It is assumed that the University specialises in teaching Arts \& Media subjects
and, as such, has no subject-specific specialist ICT requirements.

\section{Server Room}
\subsection{Physical Requirements}
The room designated to be the server room will require enough space to house all
of the equipment with around 3ft available above it for adequate ventilation
and access to any cables which may be run along the ceiling.

There will be three 42U server racks in the centre of the room allowing full access
to both the front and the back of the cabinet. It is imperative that these be
grounded correctly so that any electrical anomalies do not present a threat to
human health. These will provide adequate space for all the equipment specified
in this document in addition to plenty of space for future requirements.

The server room should be located, as far as possible, out of the way of water
and sewage services and should be accessible without climbing / descending any
staircases (lift access is acceptable). It should also be noted that it is
likely to be quite a noisy room and should be located away from areas that may
suffer as a result of loud, continuous noise. If this is not possible, an
alternative would be to sound-proof the room but it should be noted that this
can be quite expensive.

There will also need to be one rack located on the second site (see
\ref{sec:secrack}).

\subsection{Service Provision Requirements}
\label{sec:servicepro}
It is recommended that a clean and separate power supply be provided both to the
server room and, if possible, the second site's equipment rack. It is important
that other building services' fluctuating power requirements do not affect the
supply to either campus' equipment. Furthermore it is recommended that
Uninterruptable Power Supplies be provided for each rack. Many of these also
provide facilities to monitor the power consumption of individual devices (such
as those provided by Geist).

Each site will require it's own internet connection. Since this will need to
handle a lot of traffic at peak times of the day (particularly if students are
to be resident on campus during their studies), fibre optic connections are
recommended. The two sites should also be connected to each other directly by
means of another fibre optic connection to allow services provided by the main
campus to be accessed with minimal latency from the second campus.

\subsection{Other needs}
\label{sec:needs}
Server cooling will be provided by Airedale OnRak units (one per rack). These will dissipate
the heat generated and will fit into the back of each of the server racks. It is
recommended that a temperature monitor also be installed which is able to
alert relevant personel if the temperature falls outside of
normal operational ranges.

The server room must be fitted with a fire suppression system. This should use a
gasseous suppression agent such as CO\textsubscript{2} which will not damage the
sensitive equipment. This \emph{will}, however require more space to be made
available to store the suppression agent\cite{fire}.

Physical access to the room must also be kept to only those staff members who
require access to complete their tasks (and every effort should be made to
enable them to work without access to the server room). Access will also be
monitored either through a key card access system (see \ref{sec:security}) or a
physical visitor's log.

\section{Project Management}
\label{sec:projman}
The project will be overseen by a Steering Group chaired by a representative of
the University (henceforth known as the Project Client). This Steering Group
will contain a representative from all major departments of the University, a
Project Manager and the head of ICT Services.

The group will be responsible for the specification of the services to be
provided along with the smooth running and continuous day-to-day management of the project
(as overseen by the Project Manager).

\subsection{Risk Analysis}
During the planning and development phase of the project, the following risks
have been identified.

\subsubsection{Incorrect provision of hardware}
It is possible that the hardware outlined in this and future documents is
incompatible or unfit for purpose. If this is the case, new devices will be
specified and decided upon by the Steering Group.

\subsubsection{Overrunning of Budget}
While every effort has been and will continue to be made to ensure that all
financial forecasts are accurate, unforseen expenses may be incurred. As such, it
is recommended that a contingency of 10\% of the overall project cost be
included during planning which may be used to mitigate the risks of financial
over-spend.

\subsubsection{Late Delivery of Project Goals}
Each phase of the project should have an established goal and time frame within
which to be achieved. This should be closely monitored and regulated by the
Steering Group by means of a weekly status report and \emph{any} concerns should be raised as soon as possible so
that mitigating action may be decided upon and undertaken by all relevant
parties.

\subsubsection{Failure to Deliver by External Contractors}
Any phases of the project requiring external contractors are subject to further
risks as a result of the work being undertaken (e.g. work not being finished on
time or to correct specification). Therefore, any such phase should be given
extra attention by the Steering Group and any specifically identifiable risks
should be pre-emptively mitigated as far as is possible.

\subsubsection{Incorrect / Late Delivery of Hardware}
It is possible that hardware be delivered late or even damaged. Records of
\emph{all} purchases and deliveries must be kept so that any
disputes / discontinuity may be dealt with as swiftly as possible.

\section{Hardware}
\subsection{Main Campus Server Room}
The main campus server room will contain one network switch to handle external
traffic (internet and inter-site) and two further switches to distribute network
connectivity inside the server room and to the rest of the building.

\clearpage
There will also be seven individual servers providing the following functions:
\begin{itemize}
\item{Firewall server}
\item{Application server}
\item{Web server}
\item{E-mail server}
\item{Print server}
\item{Primary DNS server}
\item{Primary storage control server}
\end{itemize}
It is recommended that the exact specification of these servers be decided upon
after further requirements gathering and analysis by the Steering Group.
Consideration should be made regarding how much processing power each requires.
The most processor intensive services are likely to be provided by the
application server, the storage controller and the web server. The print and
e-mail servers are likely to need considerably less power.

In addition to this, there should be a workstation available in the server room
for system administrators to use during maintenance and operation. It is assumed
that any reasonable desktop computer will be acceptable and that some flavour of
Linux will be installed (this is such a matter of taste that it should be
decided upon by the staff who will be required to make use of it in their daily
tasks). Furthermore, provision should be made for a monitor and keyboard to be
attached to any of the servers in the racks for easy administration. Rack
mountable KVM consoles are available and may be specified by the Steering Group.

The server room will also contain a RAID disk array with at least 6TB of useable
space. This will be used to provide user's home directories and central backups
of University documents, databases etc. Again, the exact make and model will be
left to the Steering Group but it is recommended that these \emph{not} be
solid-state drives as these tend to fail with very little warning (even with
proper backups in place, the author feels that people administering the system
like to know when critical pieces of infrastructure are about to fail).

\subsection{Second Campus Hardware}
\label{sec:secrack}
The second campus will house one further 42U rack. Attention should be paid to
its placement with regard to cooling and fire safety. It should be located out
of the way but not inaccessible. This rack will contain a backup application
server, a secondary storage control server, a secondary DNS server and a
backup / testing web server. In addition to these, it will require at least one
network switch to connect to the internet and the main campus server room. There
will also be a RAID disk array identical to that on the main campus.

\subsection{User-facing Hardware}
Users on either campus should be able to connect to the network wirelessly. This
will require the provision of wireless access points through both sites. The
specification and placement of these should be decided upon by the steering
group after a comprehensive site survey.

Computer rooms, teaching rooms and offices will require workstations for the use
of staff / students / visitors. Virtually any modern desktop hardware should suffice
although it should be noted that the internal hard drive will only be required
to hold the operating system as all data storage will be performed over the
network.

It is possible that research students or other academic staff require particular
pieces of hardware to be acquired. In this case, a request should be made to the
ICT department for evaluation and processing but these requests are left
entirely at the discretion of the Head of ICT.

It is expected that both campuses will require a number of peripheral devices
(printers, scanners etc.) the location and specification of which will be
decided upon by the Steering Group. Consideration should be made, however to
ensure that they remain compatible with the rest of the system.

It is assumed that telephony requirements are being provided by a separate part
of the design specification therefore no discussion of such systems will be made
in this document.

\section{Software}
\subsection{IT \& Services Software}
All of the servers (with the exception of the firewall and network storage
covered below) will be installed with Debian Squeeze. This is an open-source
Linux distribution used on production servers worldwide\cite{debian} and is
very well supported by the community.

\subsubsection{Application Servers}
The application server will be used to provide software as a service to the
users. The following list is expected to grow as a result of further
requirements gathering activities outlined in section \ref{sec:projman}

Intacct Financial and Management Accounting software will be installed to
provide payroll, PoS, budgeting and accounting services to administrative staff
and public cafe / bar areas.

A Learning Management System will be provided by Blackboard. Blackboard has a
large market share and industry backing \cite{bb} and although open-source
alternatives were considered, none provide the range of features and support
available from Blackboard.

\subsubsection{Firewall}
One of the servers will operate as a dedicated firewall and gateway. It will be
installed with Smoothwall, a dedicated firewall distribution of Linux. It is
open-source and is well documented through its online documentation and
manuals \cite{smoothwall}.

\subsubsection{E-mail}
One of the servers will function as a dedicated e-mail server. It will be
installed with Exim as the Mail Transfer Agent and RoundCube as the webmail
front-end. Exim has been growing in popularity over recent years and is well
supported on Debian platforms\cite{exim}.

\subsubsection{Printing}
Print services will be handled by CUPS. It is likely that other utilities will
be required to be installed as part of this system but this will depend on which
printers are eventually selected and so the final decision will be in the hands
of the Steering Group.

\subsubsection{File Storage}
The disks will be presented on the network using the CIFS protocol by means of
Samba and cross-site backup policy will be enforced using rsync and cron jobs.
Samba is very well established and well documented software which is frequently
used to provide Windows and Linux home directories across the network.

\subsubsection{Naming \& Directory}
See section \ref{sec:nandd}

\subsubsection{Webserver}
One server will be given over to providing both the external website and an
internal intranet. Nginx will be installed as the webserver. Nginx has been
growing in popularity since its inception \cite{nginx}. It is open-source and
considerably less memory-intensive than Apache.

\subsection{User Software}
All student and staff workstations will be installed with Windows 7. I believe
this to be a well understood platform for the average user and it may be
possible to purchase a support contract from Microsoft.

Research students may require other versions of Windows or even Linux to be
installed in order to run legacy applications critical to their work. In this
case all software should pass through the audit procedures outlined below (see
\ref{sec:security}) before it is installed.

General office tools (word processor, spreadsheets, presentations etc.) will be
provided by LibreOffice. This is not too dissimilar from commercial offerings
and does not require licenses to be purchased.

\section{Security}
\label{sec:security}
A policy of separation of concerns (that is, logically separate systems being
kept as physically separate as possible) should be the first consideration
toward security in virtually all areas of the ICT design.

The Steering Group should also put in place a regular check of the security
systems as part of the regular maintenance schedule as mentioned below
(see \ref{sec:manage}).

\subsection{Network Security}
Network services should, where possible, be limited to only those devices or
users which require access. For example, the Point of Sale systems will only
require access to particular applications provided by the application servers
and as such should run on a separate VLAN. This means if any of the PoS
machines are compromised, the scope for damage (at least theoretically) is
limited to that sub-system of the overall network.

Furthermore, all devices wishing to access the network will be required to
register a MAC address. While it is possible to spoof MAC addresses, this adds
an extra hurdle for any potential attacker to jump.

\subsection{Application Security}
It is recommended that the Steering Group spend time putting together a document
outlining what the University considers to be important factors in ensuring
applications are safe, secure and stable enough to be deployed within the ICT
infrastructure. This should include both Software as a Service (i.e. that which
is installed on the application server and accessed remotely) \emph{and} user
software installed on individual work-stations. This document would then form
the basis of a Software Security Audit procedure to be followed when procuring new
software.

It is recommended that this procedure also be followed retro-actively for
software outlined in this document. This is not just a point of sanity, it is
intended that the results of the audits be kept on record for reference.

\subsection{Physical Security}
All workstations should be physically secured by means of a metal cable lock. It
is important to ensure that the end of the lock not secured to the machine is
attached to something large and heavy (all too often this step is skipped,
entirely negating the purpose of the exercise!).

The server room itself must be kept locked at all times and it is recommended
that some form of electronic access lock be provided which allows the monitoring
of individual's movements. In the event of a problem, it is important to know
who has accessed the room and when (see \ref{sec:needs}).

Any critical infrastructure located outside of the main server room should not
be accessible to anyone who does not require access. To this end it is
recommended that lockable (and preferable fire-proof) boxes be purchased and
located out of sight (though not inaccessible).

The second site rack should be locked at all times and should be stored in a
non-public area for increased security.

\clearpage
\section{Naming \& Directory Services}
\label{sec:nandd}
One of the servers at each site will be used exclusively as the authoritative name server for
the University's domain(s), the primary controller being located in the server
room and the secondary server on the second site. Debian will be installed as the Operating System
(this is mostly for consistency with the other servers; if the Steering Group
can justify a different OS, it should be possible to switch with little
adjustment). BIND (v9) will be used to provide DNS services across the network
and MySQL is recommended as the database controller for the records being
stored.

In addition to this, LDAP will be used to provide directory services. This will
also be used to store user data for all staff / student accounts (accounts which
will be provided with a home directory on the main filestore). It should then be
possible to use LDAP as the main authentication protocol across the network.

\section{Business Continuity}
The risks and mitigation strategies outlined below are intended to be a guide
only. Further discussion and review should be carried out by the Steering Group
(particularly with reference to specifying \emph{exactly} which facilities and
services are considered critical to the business continuity of the University).

\subsection{Risk Analysis}
\subsubsection{Catastrophic Server Room Failure}
In the event of all (or most) systems becoming unavailable in the server room
(for example as the result of fire), critical applications (identified by the
Steering Group) will be provided by the backup application server on the second
site. Network storage will be provided by the backup storage devices on the
second site although this will be considered a degraded service and only users
with business critical needs (as identified by the Steering Group) will be
allowed access. Naming and directory services will be provided by the secondary
DNS server which will also be required to handle the activities once performed
by the dedicated firewall. This too will be considered a degraded service. It is
assumed that the public website of the University is considered business
critical and as such will be served by the secondary web server in this
eventuality.

\subsubsection{Second Site Failure}
In the event of the second site's ICT facilities being compromised, services
provided by equipment on the second site should fall-back to using the primary
systems available from the main server room. Directory and naming services will
be considered a degraded service until such time as the secondary DNS server can
be reinstated. Similarly, the ability to backup data to the second site will be
unavailable. Efforts should be made to duplicate the backups available from the
server room as a precautionary measure.

\subsubsection{Loss of Inter-Site Communications}
If the connection between sites is lost, any business critical tasks which
depend heavily on services provided by the ICT department should be temporarily
re-located to the main campus and repairative action be taken immediately.

\subsubsection{Loss of Internet Communications}
If either site experiences a loss of connectivity, it should route all external
bound traffic through the other site. Based on the location of the University
and the author's own personal experience, this situation is considered likely
and adequate attention should be paid by the ICT group to ensure that this
situation is dealt with as automatically and seamlessly as possible.

\subsubsection{Network Security Breach}
In the event of a possible network security breach, the affected machines should
be brought offline as soon as possible whilst maintaining business continuity.
For services which are duplicated (in whole or in part) on the second site, the
secondary hardware and systems may be used with little to no interruption to
business continuity. If there is no backup / duplicate service available, other
hardware should be provisioned at the discretion of the Head of ICT and brought
online from backups as soon as possible. Care should be taken to ensure that any
backups used to restore offline systems be absolutely free from the
errors / concerns which caused the system to be taken offline originally.

It is possible (or even likely) that such a security breach may affect many if
not all of the systems on a single campus or even both of them. If this is the
case, web services should be moved to an external provider temporarily whilst a
full examination and rebuild of the system takes place. Planning for this sort
of eventuality is no trivial undertaking and should be discussed at length by
the Steering Group and reviewed regularly by the ICT group to ensure that
business critical services are (as far as possible) unaffected whilst still
ensuring that any concerns about the security of the failed system are not
carried over to replacement systems.

\section{Ongoing Management}
\label{sec:manage}
It is recommended that in addition to a general ICT manager, the University also
employ one or two system administrators and one ICT strategy adviser. These
should meet on a regular basis with key staff in each department (that is people
who understand what is required of the ICT systems within their own department).
This regular meeting will allow current and future aspects of the system to be
analysed and maintained without losing focus on the deliverable goals of
day-to-day operations.

In addition to this, a maintenance schedule should be drawn up consisting of the
task to be accomplished, the time it should take place and the member of staff
responsible. Regular maintenance tasks may include (but are not limited to)
updating system software, checking hard disk integrity and replacing batteries
in UPSs.

Logs of all changes / additions to the critical infrastructure should be produced
and kept up to date. These should be stored (along with other critical
documents) in the server room (preferably in a fire-proof container). This
ensures, should problems occur, that they can be traced back to their origin and
fixed much quicker and with less risk to service.

\subsection{Disaster Recovery}
Furthermore, the Steering Group should compile a disaster recovery strategy to
be maintained by the ICT committee. This will include strategies for recovery in
the event of critical system infrastructure failure. For example, if the
application server crashes, what is the procedure for switching to the back-up
server, identifying the problem and reaching a solution. These operational
procedures should be well understood by those members of the team who will be
required to act upon it and as such it is recommended that these procedures be
rehearsed at regular intervals (much the same way that a fire evacuation plan
may be rehearsed).

\subsection{Backups}
Regular backups should be made of all critical data. The following strategy is
presented as a guideline which should be revised by the Steering Group and ICT
staff.

User's home directories will be backed up incrementally overnight to the second
site. Rsync contains adequate options to ensure that this is done using minimal
network bandwidth while maintaining correct meta-data.

Critical configuration files, logs, departmental documentation and other
electronic documents identified by the University will also be backed up to the
second site; preferably more often than once per day especially for documents
which are likely to change on a regular basis (e.g. system log files). Provision
should also be made for these to be backed up in a third location entirely
off-site. Many companies are now offering cloud solutions for exactly this
situation although I believe we may be able to arrange a mutually beneficial
exchange of backups with another University / institution.

\clearpage
\section{Summary}
In conclusion, the University will require a secure server room on the main
campus which will provide the primary hardware for all systems across both
sites. It will also be necessary to house some secondary / backup hardware
securely on the second campus.

Project Management will be overseen by a Steering Group responsible for the
final specification and delivery of the services required. Control will then be
handed to the ICT department for continuous management.

Business continuity objectives will be defined and documented clearly with
well-rehearsed operational procedures for dealing with critical issues.

I believe that the infrastructure and strategies outlined in this document will
provide a comprehensive and resilient solution to the business ICT requirements
of the University at a reasonable level of cost.

\subsection{Estimated Costs}
\begin{itemize}
\item{{\bf Server Room Infrastructure} - \textsterling3,600}
\item{{\bf Server Room Services} - £\textsterling4,000}
\item{{\bf Server Room Hardware} - £\textsterling6,500}
\item{{\bf Second Campus Infrastructure} - \textsterling1,200}
\item{{\bf Second Campus Hardware} - \textsterling4,500}
\item{{\bf Network Provisioning} (e.g. site-wide ethernet) - \textsterling1,500}
\item{{\bf User Workstations} - \textsterling100,000}
\item{{\bf Software Licenses} - \textsterling75,000}
\end{itemize}
{\bf Total: £196,300}

\begin{thebibliography}{1}

\bibitem{fire} {\em
  http://www.techrepublic.com/blog/security/the-mystical-world-of-data-center-fire-suppression/4113}
\bibitem{debian} {\em http://www.debian.org/users/}
\bibitem{bb} {\em http://en.wikipedia.org/wiki/Learning\_management\_system}
\bibitem{smoothwall} {\em http://www.smoothwall.org/}
\bibitem{exim} {\em http://www.securityspace.com/s\_survey/data/man.201007/mxsurvey.html}
\bibitem{nginx} {\em http://trends.builtwith.com/Web-Server/nginx}
\bibitem{bind} {\em http://en.wikipedia.org/wiki/BIND}
\bibitem{ldap} {\em http://en.wikipedia.org/wiki/Lightweight\_Directory\_Access\_Protocol}

\end{thebibliography}

\end{document}

